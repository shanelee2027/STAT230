% Options for packages loaded elsewhere
\PassOptionsToPackage{unicode}{hyperref}
\PassOptionsToPackage{hyphens}{url}
%
\documentclass[
]{article}
\usepackage{amsmath,amssymb}
\usepackage{iftex}
\ifPDFTeX
  \usepackage[T1]{fontenc}
  \usepackage[utf8]{inputenc}
  \usepackage{textcomp} % provide euro and other symbols
\else % if luatex or xetex
  \usepackage{unicode-math} % this also loads fontspec
  \defaultfontfeatures{Scale=MatchLowercase}
  \defaultfontfeatures[\rmfamily]{Ligatures=TeX,Scale=1}
\fi
\usepackage{lmodern}
\ifPDFTeX\else
  % xetex/luatex font selection
\fi
% Use upquote if available, for straight quotes in verbatim environments
\IfFileExists{upquote.sty}{\usepackage{upquote}}{}
\IfFileExists{microtype.sty}{% use microtype if available
  \usepackage[]{microtype}
  \UseMicrotypeSet[protrusion]{basicmath} % disable protrusion for tt fonts
}{}
\makeatletter
\@ifundefined{KOMAClassName}{% if non-KOMA class
  \IfFileExists{parskip.sty}{%
    \usepackage{parskip}
  }{% else
    \setlength{\parindent}{0pt}
    \setlength{\parskip}{6pt plus 2pt minus 1pt}}
}{% if KOMA class
  \KOMAoptions{parskip=half}}
\makeatother
\usepackage{xcolor}
\usepackage[margin=1in]{geometry}
\usepackage{color}
\usepackage{fancyvrb}
\newcommand{\VerbBar}{|}
\newcommand{\VERB}{\Verb[commandchars=\\\{\}]}
\DefineVerbatimEnvironment{Highlighting}{Verbatim}{commandchars=\\\{\}}
% Add ',fontsize=\small' for more characters per line
\usepackage{framed}
\definecolor{shadecolor}{RGB}{248,248,248}
\newenvironment{Shaded}{\begin{snugshade}}{\end{snugshade}}
\newcommand{\AlertTok}[1]{\textcolor[rgb]{0.94,0.16,0.16}{#1}}
\newcommand{\AnnotationTok}[1]{\textcolor[rgb]{0.56,0.35,0.01}{\textbf{\textit{#1}}}}
\newcommand{\AttributeTok}[1]{\textcolor[rgb]{0.13,0.29,0.53}{#1}}
\newcommand{\BaseNTok}[1]{\textcolor[rgb]{0.00,0.00,0.81}{#1}}
\newcommand{\BuiltInTok}[1]{#1}
\newcommand{\CharTok}[1]{\textcolor[rgb]{0.31,0.60,0.02}{#1}}
\newcommand{\CommentTok}[1]{\textcolor[rgb]{0.56,0.35,0.01}{\textit{#1}}}
\newcommand{\CommentVarTok}[1]{\textcolor[rgb]{0.56,0.35,0.01}{\textbf{\textit{#1}}}}
\newcommand{\ConstantTok}[1]{\textcolor[rgb]{0.56,0.35,0.01}{#1}}
\newcommand{\ControlFlowTok}[1]{\textcolor[rgb]{0.13,0.29,0.53}{\textbf{#1}}}
\newcommand{\DataTypeTok}[1]{\textcolor[rgb]{0.13,0.29,0.53}{#1}}
\newcommand{\DecValTok}[1]{\textcolor[rgb]{0.00,0.00,0.81}{#1}}
\newcommand{\DocumentationTok}[1]{\textcolor[rgb]{0.56,0.35,0.01}{\textbf{\textit{#1}}}}
\newcommand{\ErrorTok}[1]{\textcolor[rgb]{0.64,0.00,0.00}{\textbf{#1}}}
\newcommand{\ExtensionTok}[1]{#1}
\newcommand{\FloatTok}[1]{\textcolor[rgb]{0.00,0.00,0.81}{#1}}
\newcommand{\FunctionTok}[1]{\textcolor[rgb]{0.13,0.29,0.53}{\textbf{#1}}}
\newcommand{\ImportTok}[1]{#1}
\newcommand{\InformationTok}[1]{\textcolor[rgb]{0.56,0.35,0.01}{\textbf{\textit{#1}}}}
\newcommand{\KeywordTok}[1]{\textcolor[rgb]{0.13,0.29,0.53}{\textbf{#1}}}
\newcommand{\NormalTok}[1]{#1}
\newcommand{\OperatorTok}[1]{\textcolor[rgb]{0.81,0.36,0.00}{\textbf{#1}}}
\newcommand{\OtherTok}[1]{\textcolor[rgb]{0.56,0.35,0.01}{#1}}
\newcommand{\PreprocessorTok}[1]{\textcolor[rgb]{0.56,0.35,0.01}{\textit{#1}}}
\newcommand{\RegionMarkerTok}[1]{#1}
\newcommand{\SpecialCharTok}[1]{\textcolor[rgb]{0.81,0.36,0.00}{\textbf{#1}}}
\newcommand{\SpecialStringTok}[1]{\textcolor[rgb]{0.31,0.60,0.02}{#1}}
\newcommand{\StringTok}[1]{\textcolor[rgb]{0.31,0.60,0.02}{#1}}
\newcommand{\VariableTok}[1]{\textcolor[rgb]{0.00,0.00,0.00}{#1}}
\newcommand{\VerbatimStringTok}[1]{\textcolor[rgb]{0.31,0.60,0.02}{#1}}
\newcommand{\WarningTok}[1]{\textcolor[rgb]{0.56,0.35,0.01}{\textbf{\textit{#1}}}}
\usepackage{graphicx}
\makeatletter
\def\maxwidth{\ifdim\Gin@nat@width>\linewidth\linewidth\else\Gin@nat@width\fi}
\def\maxheight{\ifdim\Gin@nat@height>\textheight\textheight\else\Gin@nat@height\fi}
\makeatother
% Scale images if necessary, so that they will not overflow the page
% margins by default, and it is still possible to overwrite the defaults
% using explicit options in \includegraphics[width, height, ...]{}
\setkeys{Gin}{width=\maxwidth,height=\maxheight,keepaspectratio}
% Set default figure placement to htbp
\makeatletter
\def\fps@figure{htbp}
\makeatother
\setlength{\emergencystretch}{3em} % prevent overfull lines
\providecommand{\tightlist}{%
  \setlength{\itemsep}{0pt}\setlength{\parskip}{0pt}}
\setcounter{secnumdepth}{-\maxdimen} % remove section numbering
\ifLuaTeX
  \usepackage{selnolig}  % disable illegal ligatures
\fi
\usepackage{bookmark}
\IfFileExists{xurl.sty}{\usepackage{xurl}}{} % add URL line breaks if available
\urlstyle{same}
\hypersetup{
  pdftitle={Homework 01 R Basics},
  pdfauthor={S\&DS 230/530/ENV 757},
  hidelinks,
  pdfcreator={LaTeX via pandoc}}

\title{Homework 01 R Basics}
\usepackage{etoolbox}
\makeatletter
\providecommand{\subtitle}[1]{% add subtitle to \maketitle
  \apptocmd{\@title}{\par {\large #1 \par}}{}{}
}
\makeatother
\subtitle{Due by 11:59pm, Friday, 1.24.25}
\author{S\&DS 230/530/ENV 757}
\date{}

\begin{document}
\maketitle

\begin{center}\rule{0.5\linewidth}{0.5pt}\end{center}

\textbf{(1) RMarkdown Practice} \emph{(24 points)} Change the markdown
code below as indicated.

\textbf{Make this line bold}

\emph{Make this line italics}

\subsection{Make this line a second level
header}\label{make-this-line-a-second-level-header}

\begin{itemize}
\tightlist
\item
  Make this line a bullet point

  \begin{itemize}
  \tightlist
  \item
    Make this line an indented (or level two) bullet point
  \end{itemize}
\end{itemize}

\href{https://www.nytimes.com/}{\textbf{LINK}} (make the word LINK at
left link to the New York Times home page AND make it bold)

\texttt{Make\ this\ line\ look\ like\ R\ Code}

Below this line, insert a new R chunk, create a vector called
\texttt{xvec} that contains the integers 5 through 9, and have R display
what is in \texttt{xvec}.

\begin{Shaded}
\begin{Highlighting}[]
\NormalTok{xvec }\OtherTok{\textless{}{-}} \DecValTok{5}\SpecialCharTok{:}\DecValTok{9}
\NormalTok{xvec}
\end{Highlighting}
\end{Shaded}

\begin{verbatim}
## [1] 5 6 7 8 9
\end{verbatim}

\textbf{(2) R Syntax Practice} \emph{(12 points)} Modify the R code
below to follow good R Syntax practices

\begin{Shaded}
\begin{Highlighting}[]
\NormalTok{x }\OtherTok{=} \DecValTok{5}

\NormalTok{x}\SpecialCharTok{\textless{}=}\FunctionTok{c}\NormalTok{(}\DecValTok{1}\NormalTok{, }\DecValTok{2}\NormalTok{, }\DecValTok{3}\NormalTok{)}

\FunctionTok{length}\NormalTok{ (x)}

\ControlFlowTok{for}\NormalTok{ (i }\ControlFlowTok{in} \DecValTok{1}\SpecialCharTok{:}\DecValTok{10}\NormalTok{) \{ }
\NormalTok{x }\OtherTok{\textless{}{-}}\DecValTok{1}\SpecialCharTok{+}\DecValTok{1} 
\NormalTok{           \}}

\NormalTok{x }\OtherTok{\textless{}{-}} \DecValTok{1}\NormalTok{ ; y }\OtherTok{\textless{}{-}} \FunctionTok{c}\NormalTok{(}\DecValTok{3}\NormalTok{, }\DecValTok{4}\NormalTok{)}
\end{Highlighting}
\end{Shaded}

\textbf{(3) Data handling} \emph{36 pts}

(3.1) Insert a new R code chunk below.

\begin{Shaded}
\begin{Highlighting}[]
\NormalTok{wb }\OtherTok{=} \FunctionTok{read.csv}\NormalTok{(}\StringTok{"WB\_2024.csv"}\NormalTok{)}
\FunctionTok{dim}\NormalTok{(wb)}
\end{Highlighting}
\end{Shaded}

\begin{verbatim}
## [1] 217  17
\end{verbatim}

\begin{Shaded}
\begin{Highlighting}[]
\FunctionTok{names}\NormalTok{(wb)}
\end{Highlighting}
\end{Shaded}

\begin{verbatim}
##  [1] "Country"    "Population" "Rural"      "GNI"        "Imports"   
##  [6] "Exports"    "Military"   "Cell"       "Fertility"  "Measles"   
## [11] "InfMort"    "LifeExp"    "PM2.5"      "CO2"        "EnergyUse" 
## [16] "Renewable"  "Debt"
\end{verbatim}

\begin{Shaded}
\begin{Highlighting}[]
\FunctionTok{head}\NormalTok{(wb, }\DecValTok{5}\NormalTok{)}
\end{Highlighting}
\end{Shaded}

\begin{verbatim}
##          Country Population  Rural   GNI  Imports  Exports Military      Cell
## 1    Afghanistan   42239854 73.067   360 37.06956 14.34215       NA  56.55443
## 2        Albania    2745972 35.397  6770 44.70882 31.30916 1.584881  92.31992
## 3        Algeria   45606480 24.732  4490 23.38840 23.88251 4.779438 106.42354
## 4 American Samoa      43914 12.765    NA 92.53333 44.26667       NA        NA
## 5        Andorra      80088 12.226 50080       NA       NA       NA 118.67298
##   Fertility Measles InfMort LifeExp     PM2.5         CO2 EnergyUse Renewable
## 1     4.523      68    44.8  62.879 46.087094 0.138000720      2.94      20.0
## 2     1.376      86     8.4  76.833 15.707004 1.615083618      2.27      41.9
## 3     2.829      79    18.7  77.129 25.552656 3.943578663      5.61       0.1
## 4        NA      NA      NA      NA  6.715147 0.002258713        NA       0.4
## 5        NA      98     2.6      NA  9.080281          NA      1.89      18.4
##        Debt
## 1        NA
## 2 56.302323
## 3  3.733707
## 4        NA
## 5        NA
\end{verbatim}

\begin{Shaded}
\begin{Highlighting}[]
\FunctionTok{sapply}\NormalTok{(wb, class)}
\end{Highlighting}
\end{Shaded}

\begin{verbatim}
##     Country  Population       Rural         GNI     Imports     Exports 
## "character"   "numeric"   "numeric"   "integer"   "numeric"   "numeric" 
##    Military        Cell   Fertility     Measles     InfMort     LifeExp 
##   "numeric"   "numeric"   "numeric"   "integer"   "numeric"   "numeric" 
##       PM2.5         CO2   EnergyUse   Renewable        Debt 
##   "numeric"   "numeric"   "numeric"   "numeric"   "numeric"
\end{verbatim}

\begin{Shaded}
\begin{Highlighting}[]
\CommentTok{\# GNI has class integer}
\NormalTok{wb\_Subset }\OtherTok{\textless{}{-}}\NormalTok{ wb[wb}\SpecialCharTok{$}\NormalTok{GNI }\SpecialCharTok{\textgreater{}} \DecValTok{70000} \SpecialCharTok{\&} \FunctionTok{is.na}\NormalTok{(wb}\SpecialCharTok{$}\NormalTok{GNI)}\SpecialCharTok{==}\NormalTok{F,}\FunctionTok{c}\NormalTok{(}\StringTok{"Country"}\NormalTok{, }\StringTok{"GNI"}\NormalTok{, }\StringTok{"EnergyUse"}\NormalTok{, }\StringTok{"Measles"}\NormalTok{)]}
\NormalTok{wb\_Subset}
\end{Highlighting}
\end{Shaded}

\begin{verbatim}
##           Country    GNI EnergyUse Measles
## 22        Bermuda 134640      1.52      NA
## 54        Denmark  73520      1.96      95
## 66  Faroe Islands  74420        NA      NA
## 89        Iceland  73930     12.33      91
## 94        Ireland  79730      1.09      90
## 117    Luxembourg  89200      1.98      99
## 148        Norway  96770      3.43      96
## 160         Qatar  70070      7.20      99
## 189   Switzerland  95490      1.53      96
## 207 United States  76590      4.24      92
\end{verbatim}

\begin{Shaded}
\begin{Highlighting}[]
\NormalTok{wb\_Stats }\OtherTok{\textless{}{-}} \FunctionTok{summary}\NormalTok{(wb[}\StringTok{"Debt"}\NormalTok{], }\AttributeTok{digits =} \DecValTok{2}\NormalTok{)}
\NormalTok{wb\_Stats}
\end{Highlighting}
\end{Shaded}

\begin{verbatim}
##       Debt      
##  Min.   :  2.4  
##  1st Qu.: 32.8  
##  Median : 47.4  
##  Mean   : 56.2  
##  3rd Qu.: 66.4  
##  Max.   :423.6  
##  NA's   :102
\end{verbatim}

\begin{Shaded}
\begin{Highlighting}[]
\FunctionTok{length}\NormalTok{(wb\_Stats)}
\end{Highlighting}
\end{Shaded}

\begin{verbatim}
## [1] 7
\end{verbatim}

\begin{Shaded}
\begin{Highlighting}[]
\NormalTok{wb\_Stats[}\FunctionTok{c}\NormalTok{(}\DecValTok{1}\NormalTok{, }\DecValTok{2}\NormalTok{, }\DecValTok{3}\NormalTok{, }\DecValTok{5}\NormalTok{, }\DecValTok{6}\NormalTok{)]}
\end{Highlighting}
\end{Shaded}

\begin{verbatim}
## [1] "Min.   :  2.4  " "1st Qu.: 32.8  " "Median : 47.4  " "3rd Qu.: 66.4  "
## [5] "Max.   :423.6  "
\end{verbatim}

(3.2) Read the .csv stored
\href{http://reuningscherer.net/S&DS230/data/WB_2024.csv}{HERE} into a
new data frame and call is ``wb''. This is similar to the World Bank
data I discussed in class two (this is a more current version). You can
get the longer description of each variable
\href{http://reuningscherer.net/S&DS230/data/WB_2024_NMS.csv}{HERE}

(3.3) Get the dimension of wb.

(3.4) Get the variable names of wb.

(3.5) Show the first 5 lines of wb.

(3.6) Get the data type of each variable.

(3.7) In your code, insert a comment that gives the data type of the
variable \texttt{GNI}?

(3.8) Create a new object called \texttt{wb\_Subset} that has only the
variables Country, GNI, EnergyUse, and Measles (in that order) AND only
for countries where GNI is greater than 70000. Make sure you show the
value of \texttt{wb\_Subset}.

(3.9) Get summary statistics for \texttt{Debt}. Store the results in a
new object called \texttt{wb\_Stats}. Incidentally, \texttt{wb\_Stats}
will be a vector!

(3.10) Get the length of \texttt{wb\_Stats}.

(3.11) Get r to show the following elements of \texttt{wb\_Stats} :
1,2,3,5,6 AND round the result to 1 decimal place.

\textbf{(4) Plots} \emph{16 pts}

(4.1) Using the WB dataset loaded above, make a scatterplot of ``Rural''
on the x axis and ``Measles'' on the y axis. Include a main title, axis
titles, and a non-default symbol color and symbol type. \emph{Hint:
check out ?par or see examples from class 1 or class 3}.

(4.2) Make a boxplot of the the following \texttt{wb} variables that are
all percentages: \texttt{Measles}, \texttt{Military},
\texttt{Renewable}, \texttt{Rural}. You want all four variables on one
plot. Ensure the plot has a main title, axis labels, and a unique color
for each variable.

\begin{Shaded}
\begin{Highlighting}[]
\FunctionTok{plot}\NormalTok{(wb}\SpecialCharTok{$}\NormalTok{Rural, wb}\SpecialCharTok{$}\NormalTok{Measles, }\AttributeTok{xlab=}\StringTok{"Rural"}\NormalTok{, }\AttributeTok{ylab=}\StringTok{"Measles"}\NormalTok{, }\AttributeTok{main=}\StringTok{"Scatterplot of Rural vs Measles"}\NormalTok{, }\AttributeTok{col=}\StringTok{"Blue"}\NormalTok{, }\AttributeTok{pch=}\StringTok{\textquotesingle{}x\textquotesingle{}}\NormalTok{)}
\end{Highlighting}
\end{Shaded}

\includegraphics{Homework_01_files/figure-latex/unnamed-chunk-4-1.pdf}

\begin{Shaded}
\begin{Highlighting}[]
\FunctionTok{boxplot}\NormalTok{(wb}\SpecialCharTok{$}\NormalTok{Measles, wb}\SpecialCharTok{$}\NormalTok{Military, wb}\SpecialCharTok{$}\NormalTok{Rural, wb}\SpecialCharTok{$}\NormalTok{Renewable, }
        \AttributeTok{names=}\FunctionTok{c}\NormalTok{(}\StringTok{"Measles"}\NormalTok{, }\StringTok{"Military"}\NormalTok{, }\StringTok{"Renewable"}\NormalTok{, }\StringTok{"Rural"}\NormalTok{), }
        \AttributeTok{main=}\StringTok{"Boxplot of Various Country Facts"}\NormalTok{,}
        \AttributeTok{ylab=}\StringTok{"Percentage"}\NormalTok{,}
        \AttributeTok{col=}\FunctionTok{c}\NormalTok{(}\StringTok{"Orange"}\NormalTok{, }\StringTok{"Red"}\NormalTok{, }\StringTok{"Green"}\NormalTok{, }\StringTok{"Brown"}\NormalTok{), }
        \AttributeTok{lwd=}\DecValTok{2}\NormalTok{)}
\end{Highlighting}
\end{Shaded}

\includegraphics{Homework_01_files/figure-latex/unnamed-chunk-4-2.pdf}

\textbf{(5) Lists} \emph{12 pts} The code below creates a list called
\texttt{aList}

(5.1) Compute the sum of the second element of the list's third element.
Store the result into an object named \texttt{mySum}. You'll want to use
the \texttt{sum()} function.

\begin{Shaded}
\begin{Highlighting}[]
\NormalTok{aList }\OtherTok{\textless{}{-}} \FunctionTok{list}\NormalTok{(}\FunctionTok{c}\NormalTok{(}\DecValTok{1}\NormalTok{, }\DecValTok{5}\NormalTok{, }\DecValTok{4}\NormalTok{), letters[}\FunctionTok{c}\NormalTok{(}\DecValTok{1}\NormalTok{, }\DecValTok{6}\NormalTok{, }\DecValTok{4}\NormalTok{, }\DecValTok{9}\NormalTok{, }\DecValTok{22}\NormalTok{, }\DecValTok{3}\NormalTok{)], }\FunctionTok{list}\NormalTok{(}\FunctionTok{c}\NormalTok{(}\DecValTok{1}\NormalTok{, }\DecValTok{1}\NormalTok{, }\DecValTok{1}\NormalTok{), }
                        \FunctionTok{c}\NormalTok{(}\DecValTok{14}\NormalTok{, }\DecValTok{13}\NormalTok{, }\DecValTok{12}\NormalTok{), }\FunctionTok{c}\NormalTok{(}\DecValTok{3}\NormalTok{, }\DecValTok{2}\NormalTok{, }\DecValTok{1}\NormalTok{)), }\FunctionTok{c}\NormalTok{(}\FunctionTok{runif}\NormalTok{(}\DecValTok{8}\NormalTok{)))}
\NormalTok{mySum }\OtherTok{\textless{}{-}} \FunctionTok{sum}\NormalTok{(aList[[}\DecValTok{3}\NormalTok{]][[}\DecValTok{2}\NormalTok{]])}
\NormalTok{mySum}
\end{Highlighting}
\end{Shaded}

\begin{verbatim}
## [1] 39
\end{verbatim}

(5.2) What is the difference between what is returned from the following
two commands?

\begin{Shaded}
\begin{Highlighting}[]
\NormalTok{aList[[}\DecValTok{3}\NormalTok{]][}\DecValTok{2}\NormalTok{]}
\end{Highlighting}
\end{Shaded}

\begin{verbatim}
## [[1]]
## [1] 14 13 12
\end{verbatim}

\begin{Shaded}
\begin{Highlighting}[]
\NormalTok{aList[[}\DecValTok{3}\NormalTok{]][[}\DecValTok{2}\NormalTok{]]}
\end{Highlighting}
\end{Shaded}

\begin{verbatim}
## [1] 14 13 12
\end{verbatim}

\emph{Both commands have the same first part, aList{[}{[}3{]}{]}, which
extracts the third element in aList, which is list(c(1, 1, 1), c(14, 13,
12), c(3, 2, 1)). The first command, which has {[}2{]}, returns the
second element of this list contained in a 1-element list, or list(c(14,
13, 12)). The second command, which has {[}{[}2{]}{]}, just returns the
second element of the list, which is c(14, 13, 12).}

~\\
\strut ~ ~ ~

\end{document}
